% Options for packages loaded elsewhere
\PassOptionsToPackage{unicode}{hyperref}
\PassOptionsToPackage{hyphens}{url}
\PassOptionsToPackage{dvipsnames,svgnames,x11names}{xcolor}
%
\documentclass[
  20pt,
  number,
  preprint,
  3p,
  twocolumn]{elsarticle}

\usepackage{amsmath,amssymb}
\usepackage{iftex}
\ifPDFTeX
  \usepackage[T1]{fontenc}
  \usepackage[utf8]{inputenc}
  \usepackage{textcomp} % provide euro and other symbols
\else % if luatex or xetex
  \usepackage{unicode-math}
  \defaultfontfeatures{Scale=MatchLowercase}
  \defaultfontfeatures[\rmfamily]{Ligatures=TeX,Scale=1}
\fi
\usepackage[]{libertinus}
\ifPDFTeX\else  
    % xetex/luatex font selection
\fi
% Use upquote if available, for straight quotes in verbatim environments
\IfFileExists{upquote.sty}{\usepackage{upquote}}{}
\IfFileExists{microtype.sty}{% use microtype if available
  \usepackage[]{microtype}
  \UseMicrotypeSet[protrusion]{basicmath} % disable protrusion for tt fonts
}{}
\makeatletter
\@ifundefined{KOMAClassName}{% if non-KOMA class
  \IfFileExists{parskip.sty}{%
    \usepackage{parskip}
  }{% else
    \setlength{\parindent}{0pt}
    \setlength{\parskip}{6pt plus 2pt minus 1pt}}
}{% if KOMA class
  \KOMAoptions{parskip=half}}
\makeatother
\usepackage{xcolor}
\setlength{\emergencystretch}{3em} % prevent overfull lines
\setcounter{secnumdepth}{3}
% Make \paragraph and \subparagraph free-standing
\ifx\paragraph\undefined\else
  \let\oldparagraph\paragraph
  \renewcommand{\paragraph}[1]{\oldparagraph{#1}\mbox{}}
\fi
\ifx\subparagraph\undefined\else
  \let\oldsubparagraph\subparagraph
  \renewcommand{\subparagraph}[1]{\oldsubparagraph{#1}\mbox{}}
\fi


\providecommand{\tightlist}{%
  \setlength{\itemsep}{0pt}\setlength{\parskip}{0pt}}\usepackage{longtable,booktabs,array}
\usepackage{calc} % for calculating minipage widths
% Correct order of tables after \paragraph or \subparagraph
\usepackage{etoolbox}
\makeatletter
\patchcmd\longtable{\par}{\if@noskipsec\mbox{}\fi\par}{}{}
\makeatother
% Allow footnotes in longtable head/foot
\IfFileExists{footnotehyper.sty}{\usepackage{footnotehyper}}{\usepackage{footnote}}
\makesavenoteenv{longtable}
\usepackage{graphicx}
\makeatletter
\def\maxwidth{\ifdim\Gin@nat@width>\linewidth\linewidth\else\Gin@nat@width\fi}
\def\maxheight{\ifdim\Gin@nat@height>\textheight\textheight\else\Gin@nat@height\fi}
\makeatother
% Scale images if necessary, so that they will not overflow the page
% margins by default, and it is still possible to overwrite the defaults
% using explicit options in \includegraphics[width, height, ...]{}
\setkeys{Gin}{width=\maxwidth,height=\maxheight,keepaspectratio}
% Set default figure placement to htbp
\makeatletter
\def\fps@figure{htbp}
\makeatother

\newpageafter{author}
\makeatletter
\makeatother
\makeatletter
\makeatother
\makeatletter
\@ifpackageloaded{caption}{}{\usepackage{caption}}
\AtBeginDocument{%
\ifdefined\contentsname
  \renewcommand*\contentsname{Table of contents}
\else
  \newcommand\contentsname{Table of contents}
\fi
\ifdefined\listfigurename
  \renewcommand*\listfigurename{List of Figures}
\else
  \newcommand\listfigurename{List of Figures}
\fi
\ifdefined\listtablename
  \renewcommand*\listtablename{List of Tables}
\else
  \newcommand\listtablename{List of Tables}
\fi
\ifdefined\figurename
  \renewcommand*\figurename{Figure}
\else
  \newcommand\figurename{Figure}
\fi
\ifdefined\tablename
  \renewcommand*\tablename{Table}
\else
  \newcommand\tablename{Table}
\fi
}
\@ifpackageloaded{float}{}{\usepackage{float}}
\floatstyle{ruled}
\@ifundefined{c@chapter}{\newfloat{codelisting}{h}{lop}}{\newfloat{codelisting}{h}{lop}[chapter]}
\floatname{codelisting}{Listing}
\newcommand*\listoflistings{\listof{codelisting}{List of Listings}}
\makeatother
\makeatletter
\@ifpackageloaded{caption}{}{\usepackage{caption}}
\@ifpackageloaded{subcaption}{}{\usepackage{subcaption}}
\makeatother
\makeatletter
\@ifpackageloaded{tcolorbox}{}{\usepackage[skins,breakable]{tcolorbox}}
\makeatother
\makeatletter
\@ifundefined{shadecolor}{\definecolor{shadecolor}{rgb}{.97, .97, .97}}
\makeatother
\makeatletter
\makeatother
\makeatletter
\makeatother
\usepackage{float}
\makeatletter
\let\oldlt\longtable
\let\endoldlt\endlongtable
\def\longtable{\@ifnextchar[\longtable@i \longtable@ii}
\def\longtable@i[#1]{\begin{figure}[H]
\onecolumn
\begin{minipage}{0.5\textwidth}
\oldlt[#1]
}
\def\longtable@ii{\begin{figure}[H]
\onecolumn
\begin{minipage}{0.5\textwidth}
\oldlt
}
\def\endlongtable{\endoldlt
\end{minipage}
\twocolumn
\end{figure}}
\makeatother
\ifLuaTeX
  \usepackage{selnolig}  % disable illegal ligatures
\fi
\usepackage[]{natbib}
\bibliographystyle{unsrt}
\IfFileExists{bookmark.sty}{\usepackage{bookmark}}{\usepackage{hyperref}}
\IfFileExists{xurl.sty}{\usepackage{xurl}}{} % add URL line breaks if available
\urlstyle{same} % disable monospaced font for URLs
\hypersetup{
  pdftitle={The Hierarchical Structure and Longitudinal Measurement Invariance of Externalizing Symptoms in the Adolescent Brain and Cognitive Development (ABCD©) Study},
  pdfauthor={Colin Vize; Whitney R. Ringwald; Emily R. Perkins; Rebecca E. Waller; Samuel W. Hawes; Amy L. Byrd},
  pdfkeywords={HiTOP, externalizing, longitudinal invariance, youth
psychopathology, ABCD Study},
  colorlinks=true,
  linkcolor={blue},
  filecolor={Maroon},
  citecolor={Blue},
  urlcolor={Blue},
  pdfcreator={LaTeX via pandoc}}

\setlength{\parindent}{6pt}
\begin{document}

\begin{frontmatter}
\title{The Hierarchical Structure and Longitudinal Measurement
Invariance of Externalizing Symptoms in the Adolescent Brain and
Cognitive Development (ABCD©) Study \\\large{Hierarchical Structure and
Measurement Invariance of Externalizing Symptoms} }
\author[1]{Colin Vize%
\corref{cor1}%
}
 \ead{cev18@pitt.edu} 
\author[2]{Whitney R. Ringwald%
%
}
 \ead{wringwald@pitt.edu} 
\author[3]{Emily R. Perkins%
%
}
 \ead{perkinse@sas.upenn.edu} 
\author[4]{Rebecca E. Waller%
%
}
 \ead{rwaller@sas.upenn.edu} 
\author[5]{Samuel W. Hawes%
%
}
 \ead{shawes@fiu.edu} 
\author[6]{Amy L. Byrd%
%
}
 \ead{byrdal@upmc.edu} 

\affiliation[1]{organization={University of
Pittsburgh, Psychology},addressline={4113 Sennott
Square},city={Pittsburgh},country={USA},countrysep={,},postcode={15260},postcodesep={}}
\affiliation[2]{organization={University of
Pittsburgh, Psychology},city={Pittsburgh},country={USA},countrysep={,},postcodesep={}}
\affiliation[3]{organization={University of
Pennsylvania, Psychology},city={Philadelphia},country={USA},countrysep={,},postcodesep={}}
\affiliation[4]{organization={University of
Pennsylvania, Psychology},city={PA},country={USA},countrysep={,},postcodesep={}}
\affiliation[5]{organization={Florida International University, Center
for Children and
Families},city={Miami},country={USA},countrysep={,},postcodesep={}}
\affiliation[6]{organization={University of Pittsburgh Medical
Center, Psychiatry},city={Pittsburgh},country={USA},countrysep={,},postcodesep={}}

\cortext[cor1]{Corresponding author}






        
\begin{abstract}
Recent years have seen a greater shift towards alternative nosological
frameworks of psychopathology, which frequently include a dimension of
externalizing psychopathology. The Hierarchical Taxonomy of
Psychopathology (HiTOP) is one such framework. However, the HiTOP has
been most often studied in adults and more research is needed in order
to better understand the similarities and differences in the structure
of externalizing psychopathology earlier in development. This
preregistered study sought to examine the validity and utility of
extending the HiTOP externalizing dimension and its subdimensions to
youth using longitudinal data from the Adolescent Brain and Cognitive
Development (ABCD) Study. There were two primary aims: Aim 1: Identify
the hierarchical structure of externalizing psychopathology and examine
evidence of discriminant validity of the identified dimensions; and Aim
2: Assess the longitudinal measurement invariance of a broad
externalizing dimension in the ABCD study, as well as specific
underlying symptom dimensions. Results for Aim 1 analyses identified a
coherent factor structure comprising three dimensions (narrow
externalizing, irritability, and neurodevelopmental problems), and these
factors showed important similarities and differences in relation to
external correlates. Aim 2 analyses showed that strong invariance was
supported for the narrow externalizing and irritability dimensions,
while partial strong invariance was supported for broad externalizing
and neurodevelopmental problems. Quantification of measurement
(non)invariance revealed small effect sizes. Collectively, these results
highlight important directions for future research on the HiTOP model in
the ABCD study and other youth samples.
\end{abstract}





\begin{keyword}
    HiTOP \sep externalizing \sep longitudinal invariance \sep youth
psychopathology \sep 
    ABCD Study
\end{keyword}
\end{frontmatter}
    \ifdefined\Shaded\renewenvironment{Shaded}{\begin{tcolorbox}[sharp corners, borderline west={3pt}{0pt}{shadecolor}, enhanced, interior hidden, boxrule=0pt, breakable, frame hidden]}{\end{tcolorbox}}\fi

In recent years, researchers have increasingly pursued alternative
nosological frameworks of psychopathology with the goal of enhancing
etiological models and, ultimately, improving prevention and
intervention efforts \citep{eaton2023}. One popular framework, the
Hierarchical Taxonomy of Psychopathology {[}({]}HiTOP;
\citep{kotov2017}{]}, has generated significant research interest. The
HiTOP aims to establish an empirically derived structure of
psychopathology using techniques, such as factor analysis to identify
transdiagnostic dimensions that explain covariation among observed
symptoms. Importantly, the hierarchical organization of HiTOP
incorporates general and specific symptoms of psychopathology, thereby
avoiding the limitations of categorical approaches like that of the
Diagnostic and Statistical Manual of Mental Disorders {[}DSM-5;
\citep{apa2013}{]}, including high rates of within-disorder
heterogeneity and comorbidity \citep{conway2022}. Thus, the HiTOP aims
to generate a nosological framework that is empirically valid and
capable of benefiting both researchers and clinicians \citep{kotov2021}.
Indeed, across a range of mental health problems, dimensional
assessments produce greater validity and reliability compared to
categorical assessments (e.g., Markon et al., 2011) and emerging
evidence suggests that clinicians find more utility in transdiagnostic,
dimensional approaches like HiTOP compared to traditional categorical
approaches \citep[e.g.,][]{balling2023}.

However, the majority of research examining HiTOP has been conducted in
adults, leaving notable gaps in our understanding of how it can be
extended to youth during sensitive developmental windows. The present
study examined whether components of the HiTOP model, specifically its
externalizing dimensions, could be identified in the Adolescent Brain
and Cognitive Development (ABCD) Study, a longitudinal, nationally
representative study of youth in the United States. The longitudinal
nature of the ABCD study allows for tests of whether externalizing
dimensions are psychometrically invariant over time---an essential
precursor to more advanced modeling techniques (e.g., latent growth
models) that can be used to study trajectories of externalizing
dimensions over development. Addressing these gaps has the potential to
expand the relevance of HiTOP to earlier periods of development, a
critical step to further advance HiTOP-focused research
\citep{tackett2022}.

\hypertarget{hitop-as-a-model-of-externalizing-psychopathology}{%
\section{HiTOP as a Model of Externalizing
Psychopathology}\label{hitop-as-a-model-of-externalizing-psychopathology}}

The HiTOP is organized along continuous dimensions, consistent with
evidence showing that differences in psychopathology are a matter of
degree as opposed to kind \citep{haslam2012}. The HiTOP is also
hierarchical, reflecting decades of research establishing that symptoms
can be organized into broader dimensions, which in turn can advance
research on shared and distinct etiological mechanisms of
psychopathology symptoms \citep{lahey2017}. At the finest level, the
HiTOP provides descriptions of narrow behavioral (e.g., aggression,
inattention) and interpersonal (e.g., callousness) symptoms or traits.
One can move up the hierarchy to broader dimensions that are called
superspectra in the HiTOP (e.g., externalizing and internalizing
psychopathology). This dimensional, hierarchical approach allows
researchers to study features of psychopathology at varying levels of
specificity.

In adults, the externalizing superspectrum encompasses two lower order
dimensions (termed spectra in the HiTOP): disinhibited externalizing and
antagonistic externalizing \citep{krueger2021}. The disinhibited
externalizing spectrum includes the tendency to act impulsively, with
little regard for future consequences. The antagonistic externalizing
spectrum reflects tendencies to navigate interpersonal situations with
little concern for others, including a willingness to exploit and
deceive others. The broader externalizing dimension has been linked to
various maladaptive outcomes, including aggression, antisocial behavior,
and problematic substance use \citep[e.g.,][]{krueger2007}, while the
lower order spectra, including disinhibition and antagonism, offer
greater specificity and show differential associations with external
correlates \citep{mullins-sweatt2022}. For example, antagonism is more
strongly related to various forms of aggression and antisocial behavior
relative to disinhibitory symptoms \citep{vize2018}, while disinhibitory
symptoms are more closely related to problematic substance use
\citep[e.g.,][]{venables2012}. While there is substantial evidence
supporting the presence of a broad externalizing dimension in youth
\citep[e.g.,][]{bongers2004}, less work has examined whether the
antagonistic and disinhibited HiTOP dimensions can be similarly captured
in youth samples and if these spectra show unique associations with
relevant clinical and psychosocial correlates.

\hypertarget{hierarchical-models-of-externalizing-psychopathology-in-youth}{%
\section{Hierarchical Models of Externalizing Psychopathology in
Youth}\label{hierarchical-models-of-externalizing-psychopathology-in-youth}}

While the HiTOP has seen less direct application in youth samples, there
is a rich history of dimensional and hierarchical models of
psychopathology in youth. For example, models of externalizing
psychopathology have been examined across a wide range of developmental
periods, including preschool \citep[e.g.,][]{hill2006} and early
adolescence \citep[e.g.,][]{goulter2022}. Across these studies, results
have consistently shown that a broadband externalizing factor can be
reliably assessed from more fine-grained indicators and the
externalizing factor is reliably linked to specific problem behaviors
(e.g., aggression and rule-breaking). Notably, this work used a wide
variety of indicators to model higher-order externalizing dimensions,
including DSM-based diagnostic symptoms \citep[e.g.,][]{king2018},
personality features \citep[e.g.,][]{tackett2014}, and
biologically-based indicators like salivary cortisol levels
\citep[e.g.,][]{shirtcliff2005}. Other relevant examples of
externalizing-related research in youth involve research on hierarchical
models of maladaptive personality that include antagonistic and
disinhibitory personality features, which have also found strong support
in youth samples \citep[e.g.,][]{fruyt2014}. Taken together, this work
suggests that dimensional, hierarchical approaches like HiTOP can be
successfully applied to youth populations---and, moreover, that HiTOP
can serve as a unifying framework for many existing lines of
developmental psychopathology research.

Along these lines, more recent work has sought to explicitly examine the
HiTOP model in youth samples. For example, using data from the baseline
assessment of the Adolescent Brain Cognitive Development (ABCD)TM study,
\citep{michelini2019} conducted a ``bass-ackwards'' factor analysis
\citep{goldberg2006} using 102 items of the parent-report Child Behavior
Checklist \citep{achenbach1999}. At the most specific level of the
hierarchy, the authors identified five factors that they labeled
externalizing, neurodevelopmental, internalizing, somatoform, and
detachment. The authors noted that the factors mostly align with the
structure of psychopathology as detailed in HiTOP, with the exception
that the neurodevelopmental factor that emerged from the broader
externalizing factor is not included in the HiTOP model. In a more
recent effort, \citep{forbes2023} pooled youth self-report data across
various early- and mid-adolescent samples (age range=6-18 years old;
N=18,290) to identify similarities and differences between the
hierarchical structure of symptoms in youth and the HiTOP structure in
adults. The broad factors identified were termed externalizing, eating
pathology, uncontrollable worry/obsessions and compulsions, and
internalizing. Additionally, more specific dimensions of externalizing
were noted, including impulsive anger, antagonism, retaliatory anger,
and positive psychosis. While the antagonism dimension mirrored the
antagonistic externalizing dimension in adults (i.e., consisting of
items indexing aggression and cruelty towards others), no clear
disinhibited externalizing factor emerged. Instead, items reflecting
disinhibitory content (e.g., ``Generally, I am an impulsive person'')
showed primary loadings on the impulsive anger dimension. Thus, existing
research has highlighted similarities and differences between the HiTOP
model among adults and youth. Further research is needed to further
clarify how the HiTOP could be extended to youth.

\hypertarget{longitudinal-invariance-of-externalizing-psychopathology-in-youth}{%
\section{Longitudinal Invariance of Externalizing Psychopathology in
Youth}\label{longitudinal-invariance-of-externalizing-psychopathology-in-youth}}

In addition to characterizing the hierarchical structure of
externalizing psychopathology in youth, it is important to determine
whether this structure is consistent over time \citep{vandenberg2000}.
Specifically, examining the longitudinal measurement invariance of
externalizing psychopathology, or the degree to which the measurement of
this latent construct is psychometrically equivalent over time, can help
establish whether observed changes reflect true changes. Because the
developmentally relevant behavioral and interpersonal manifestations of
externalizing psychopathology are known to change across development
\citep{petersen2020}, establishing longitudinal measurement invariance
is necessary to ensure that observed changes are reflective of changes
in externalizing rather than developmentally limited indicators. For
example, decreases in temper tantrums are likely to reflect an
independent developmental process separate from true decreases in a
latent externalizing propensity. Longitudinal invariance is particularly
important in the ABCD study, given its unique potential to serve as a
scientific resource as a nationally representative, longitudinal study
of youth and their families. The ABCD study is ideally situated to
provide data that can better characterize risk and resilience factors
for within-individual change in externalizing psychopathology. Thus, the
transparent development of longitudinally invariant measurement models
of externalizing dimensions is essential.

\hypertarget{current-study}{%
\section{Current Study}\label{current-study}}

The current preregistered study sought to build on recent efforts to
extend the HiTOP model to developmental samples
\citep[e.g.,][]{forbes2023}, while using a large, nationally
representative sample of youth followed over multiple years. The current
study had two aims: \textbf{Aim 1}: Identify the hierarchical structure
of externalizing psychopathology and examine evidence of discriminant
validity of the identified dimensions; and \textbf{Aim 2}: Assess the
longitudinal measurement invariance of a broad externalizing dimension
in the ABCD study as well as more specific dimensions underlying broad
externalizing symptoms. Collectively, the current study aimed to
investigate the validity and utility of disaggregating the externalizing
dimension specifically in youth during a developmentally sensitive
window.

\hypertarget{method}{%
\section{Method}\label{method}}

\textbf{Transparency and Openness} The analysis plan was preregistered
and can be found at
\href{https://osf.io/pqdfa/?view_only=a51ffc005e4f41daac4a3b804d23ddb2}{this
link}. Deviations from our preregistered plan are reported in the
supplement. R and MPlus code to reproduce our analyses are available at
\href{https://osf.io/ec36x/?view_only=d63a1452f133421b9e1bef34a41675f1}{this
link}. Data are not posted to OSF because ABCD data are restricted to
researchers with approved access. However, the code posted to OSF
denotes the data files used and how they were cleaned for our analyses,
making the results reproducible for researchers with access to ABCD
data.

\hypertarget{sample}{%
\section{Sample}\label{sample}}

Data were drawn from four waves (baseline, 1-, 2-, and 3-year follow-up)
of the Adolescent Brain Cognitive Development (ABCD)TM study (N=11,875
at baseline; Mage=9.51; 48\% girls; 57\% White; 15\% Black; 20\%
Hispanic/Latino/a). For our primary measures, sample sizes ranged from
N=11,862 at baseline to N=10,099 at the 3-year follow-up assessment. A
more complete description of the ABCD study, including previous
publications using the data, is available at the {[}ABCD Study site{]}
(https://abcdstudy.org/).

\hypertarget{measures}{%
\section{Measures}\label{measures}}

\textbf{Externalizing Dimensions (Parent-Report: T1-T4)}. Externalizing
psychopathology was assessed using a subset of items from the
parent-report Child Behavior Checklist {[}CBCL;
\citep{achenbach1999}{]}, which was administered at all four
assessments. Specifically, we used the 45 CBCL items and item composites
identified in factor analyses of the ABCD baseline data by
\citep{michelini2019} that had primary loadings on the externalizing
factor and/or the neurodevelopmental factor. These two factors had a
total of 45 items/item composites that showed either primary loadings on
one of the factors or loaded on both factors without a clear primary
loading (e.g., the item ``impulsive or acts without thinking'' had a
loading of .49 on both the externalizing and neurodevelopmental
factors). The preregistration provides details for the specific CBCL
items that were used for our analyses.

\textbf{Diagnostic Symptom Counts (Parent- and Youth-Report; T1)}.
Symptom counts were assessed using clinician ratings of parent- and
youth-reported modules of the Kiddie Schedule for Affective Disorders
and Schizophrenia {[}KSADS-5; \citep{kobak2013}{]}. Dimensional symptom
counts for the following diagnostic constructs were used: conduct
disorder (CD), oppositional defiant disorder (ODD), attention deficit
hyperactivity disorder (ADHD), major depressive disorder,
suicidality/self-harm, generalized anxiety disorder, and social anxiety
disorder. Each symptom was indicated as being present (1) or not present
(0), and symptoms were summed such that higher scores indicate more
symptoms endorsed for the diagnostic construct. Parents completed all
KSADS-5 modules at the baseline assessment, while youth only completed
mood disorder, social anxiety, generalized anxiety disorder,
suicidality, and sleep modules at baseline.

\textbf{Prosocial Behavior (Parent- and Youth-Report; T1)}. Prosocial
behavior was assessed using the prosocial scale of the Strengths and
Difficulties Questionnaire (SDQ), which included 3 items (``try to be
nice to other people''; ``care about their feelings''; ``offer to help
others''). Items were rated on a 3-point scale ranging from 0 (not true)
to 2 (certainly true) and summed such that higher scores represent
greater levels of prosocial behavior.

\textbf{Impulsivity (Youth-Report; T1)}. Impulsivity was assessed using
the youth-reported Urgency, Premeditation (lack of), Perseverance (lack
of), Sensation Seeking, Positive Urgency, Impulsive Behavior Scale
(UPPS-P for Children Short Form -- ABCD version). The scale had 20 items
assessing impulsivity (e.g., ``I like to stop and think about things
before I do them (reversed)'') and includes 5 subscales: negative
urgency, positive urgency, lack of perseverance, lack of planning, and
sensation seeking. Items were rated on a 4-point scale from 1 (not at
all like me) to 4 (very much like me) and summed such that higher scores
represent higher impulsivity.

\textbf{Fluid Intelligence} Composite (T1). Fluid intelligence was
assessed using an age-corrected composite of 5 tasks from the NIH
Toolbox Cognition measures \citep[see][ for a detailed
description]{luciana2018}: List Sorting Working Memory Test, Pattern
Comparison Processing Speed Test, Picture Sequence Memory Test, Flanker
Task, and Dimensional Change Card Sort Test. These tasks collectively
assess abilities related to processing speed, episodic memory, working
memory, cognitive control, and cognitive flexibility.

\hypertarget{preregistered-analyses-and-hypotheses}{%
\section{Preregistered Analyses and
Hypotheses}\label{preregistered-analyses-and-hypotheses}}

\textbf{Aim 1a Analyses}: We used parallel analysis and the minimum
average partial correlation (MAP) to estimate the number of factors to
extract from the 45 items/composites. Next, we used \citep{forbes2023}
recently developed extension to Goldberg's ``bass-ackwards'' analysis
\citep{goldberg2006} to identify the hierarchical structure of
externalizing items/composites using the minimum residual factor
extraction method and promax rotation. \citep{forbes2023} extended
bass-ackwards approach differs from the traditional approach in that it
1) identifies redundant components that perpetuate through multiple
levels of the hierarchy; 2) aids in identification of artifactual
components; and 3) plots the strongest factor correlations among the
remaining factors to identify their hierarchical structure. Although
past work has used similar factor analytic approaches on CBCL data in
the ABCD sample \citep[e.g.,][]{michelini2019}, by constraining our
analyses to only focus on the 45 CBCL items/composites associated with
broad externalizing, we expected more fine-grained differences to emerge
in the hierarchical structure of the CBCL items/composites. \textbf{Aim
1b Analyses}: After identifying the hierarchical structure of
externalizing dimensions, we assessed evidence discriminant validity by
examining bivariate and semipartial correlations between the identified
externalizing dimensions (assessed using factor scores) and external
criterion measures (e.g., diagnostic symptom counts, prosocial behavior,
impulsivity, fluid intelligence). Correlations for specific dimensions
were also compared to one another using tests of dependent correlations
to detect significant differences between correlations.

\textbf{Aim 2a Analyses}: Next, we examined the longitudinal measurement
invariance of each externalizing dimension through a series of
confirmatory factor analyses (CFA). Longitudinal measurement invariance
was examined in typical fashion moving from configural to strong
invariance. In some cases, we also examined strict invariance. To test
this, we created item parcels from the CBCL items based on a set of
preregistered criteria. The item parcels served as the observed
indicators of externalizing dimensions for all tests of longitudinal
measurement invariance. \textbf{Aim 2b Analyses}: To quantify the degree
of invariance, we computed Cohen's d for mean and covariance structures
{[}dMACS; \citep{nye2011}{]}. These metrics index the collective impact
of loading and intercept noninvariance in the metric of Cohen's d, with
higher values indicating a greater degree of indicator noninvariance.

\hypertarget{preregistered-hypotheses}{%
\section{Preregistered Hypotheses}\label{preregistered-hypotheses}}

We had three primary hypotheses. Hypothesis 1a: We hypothesized that
disinhibition and antagonism dimensions would be identified at more
fine-grained levels of the hierarchy with the same items that have been
shown to comprise a broad externalizing dimension. If parallel analysis
and the minimum average partial correlation test (MAP) suggested a
relatively large number of factors could be extracted from the CBCL
items (e.g., 5 or more), we expected that the disinhibition and
antagonism factors would emerge relatively early in the factor
extraction process (e.g., at level 2 or level 3). Hypothesis 1b: We
expected evidence of discriminant findings for the derived disinhibition
and antagonism factor scores derived. Specifically, we hypothesized that
the disinhibition factor would have larger (+) associations with ADHD,
impulsivity, and fluid intelligence and the antagonism factor would have
larger (-) associations with CD, ODD, and prosocial behavior.
Furthermore, we expected these correlations to be significantly stronger
than the correlations disinhibition and antagonism showed with
internalizing outcomes. Hypothesis 2: We would be able to establish
longitudinal strong measurement invariance for the measurement models of
externalizing factors, facilitating future investigations of mean-level
change over time in these externalizing factors in the ABCD study.

\hypertarget{results}{%
\section{Results}\label{results}}

\textbf{Aim 1a: Hierarchical structure of externalizing psychopathology}
Results of the parallel analysis based on polychoric correlations among
the CBCL items suggested up to 14 factors could be extracted from the 45
CBCL items, while MAP suggested four factors for extraction. Consistent
with our preregistered criteria for factor extraction, we interpreted
the 1-, 2-, 3- and 4-factor solutions. However, the fourth factor of the
solution was not substantively meaningful, and we thus focus on the
3-factor hierarchy.

The first level of the bass-ackwards hierarchy was a broad externalizing
factor with all 45 CBCL items showing moderately strong to strong
standardized loadings on the factor (range=.47-.81). At the second
level, a neurodevelopmental problems factor emerged, characterized by
distractibility, poor motor coordination, and hyperactivity. The broad
externalizing factor remained virtually unchanged and was strongly
correlated with the broad externalizing factor from level 1 of the
hierarchy (r=.96). At level 3, the neurodevelopmental problems factor
remained unchanged (r=1.00 with the level 2 factor). However, the broad
externalizing factor split into a narrower externalizing and an
irritability factor, both of which were strongly related to the broad
externalizing factor (rs=.91). The narrow externalizing factor was
characterized by aggression (e.g., threatening others, fighting),
meanness to others (e.g., cruelty, bullying, lack of guilt), and
rule-breaking behaviors (e.g., stealing, destroys composite), and the
irritability factor was indexed by items related to affective reactivity
and lability, hostility, and distrust of others. These three levels of
the hierarchy are displayed in Figure 1. The extracted factor scores
from level 3 were strongly related to the broad externalizing factor
score (r range=.80-.87) from level 1 and positively related to one
another (r range =.52-.62). Aim 1b: Discriminant validity of the
identified dimensions

We extracted scores for the factors identified at level 3 of the
bass-ackwards hierarchy: narrow externalizing, irritability, and
neurodevelopmental problems. Factor scores were also extracted for broad
externalizing (i.e., externalizing at level 1 of the bass-ackwards
hierarchy). Correlations between factor scores and external criterion
measures are presented in \textbf{Table 1}.

\emph{Diagnostic Symptom Counts (Parent- and Youth-Report)}. At the
bivariate level, all factors were positively associated with symptom
counts across all disorders and informants, and associations were
stronger for parent-reported symptoms (r range=.07-.66) compared to
youth-reported symptoms (r range=.04-.16). Furthermore, these factors
tended to be more strongly related to symptoms of CD, ODD, and ADHD (r
range=.31-.60) compared to symptoms of depression and anxiety (r
range=.04-.32).

Semipartial correlations accentuated differences observed at the
bivariate level, most prominently for parent-report, providing strong
evidence for discriminant validity. Compared to other factors, narrow
externalizing showed the strongest associations with CD symptoms
(sr=.59) and the irritability factor with ODD symptoms (sr=.43). The
neurodevelopmental problems factor was unrelated to CD (sr=-.02) and ODD
(sr=-.04) symptoms, and strongly associated with ADHD symptoms (sr=.50).
Associations with depression and anxiety symptoms were notably smaller,
and after accounting for overlap, the narrow externalizing factor
demonstrated a negative association with generalized (sr=-.12) and
social (sr=-.05) anxiety disorder symptoms. The irritability factor
showed the strongest association with depression (sr=.20) and
generalized anxiety symptoms (sr=.21) and the neurodevelopmental
problems factor with social anxiety disorder (sr=.13). Notably, there
was little difference across the three factors in the strength of
associations with suicidality/self-harm (sr range=.08-.11).

\emph{Prosocial Behavior (Parent- and Youth-Report)}. At the bivariate
level, all factors were negatively associated with prosocial behavior,
and these associations were stronger for parent-report (r range=-.22 to
-.35) compared to youth-report (r range=-.06 to -.09). Semipartial
correlations were smaller for narrow externalizing (sr=-.21) and
irritability (sr=-.18), and reduced to non-significance for
neurodevelopmental problems (sr=-.01).

\emph{Impulsivity}. All factors showed similarly sized positive
associations to all indices of impulsivity, though these were notably
small in magnitude (r=.04-.16). The narrow externalizing and
irritability factors showed the strongest associations with negative
urgency (.16 and .15, respectively), and the neurodevelopmental problems
factor was most strongly associated with lack of perseverance (r=.18).
Semipartial correlations were smaller in magnitude, but similar to one
another. The exception was the relation between neurodevelopmental
problems and lack of perseverance (sr=.18), compared to the relation to
the other factors (sr range=-.02-.03).

\emph{Fluid Intelligence}. All factors were negatively associated with
fluid intelligence, though these associations were small in magnitude.
The neurodevelopmental problems factor demonstrated the strongest
negative association (r=-.17), followed by the narrow externalizing
factor (r=-.14), and the irritability factor (r=-.08). Semipartial
correlations were similar, though the relation between irritability and
fluid intelligence became positive (sr=.08).

\textbf{Aim 2a: Longitudinal Measurement Invariance} All items were
randomly assigned to parcels, and the parcels were used as indicators of
a given latent factor at baseline, one-year follow-up, two-year
follow-up, and three-year follow-up assessments. \textbf{Supplementary
Table S1} provides details on the number of items assigned to each
parcel and item content for each parcel. \textbf{Supplementary Tables
S2-S5} provide descriptive statistics for the parcel indicators for each
factor. \textbf{Table 2} provides results for all tests of longitudinal
measurement invariance. All models of the broad externalizing factor and
the three narrower factors showed excellent absolute fit across all
invariance tests, and strong standardized factor loadings for the
parcels (λs=.58-.89) in the configural models, as well as strong
stability over time (rs=.68-.81). Model comparison results are
summarized below (see Table 2).

\emph{Broad Externalizing Factor}. Relative to the configural model,
restricting factor loadings to be equal (i.e., test of weak invariance)
did not decrement model fit (ΔCFI=.001; ΔNCI=.005). However,
constraining the parcel indicator intercepts to be equal across time
(i.e., the test of strong invariance) did result in a significant
decrement in model fit (ΔCFI=.004; ΔNCI=.026). In an attempt to
establish partial strong invariance, we freed the intercept constraints
on parcels 4 and 5 based on modification indices. This model
demonstrated excellent overall fit (χ\^{}2=947.21; df=152; RMSEA=.021;
CFI=.994; TLI=.992) and showed no decrement in model fit compared to the
weak invariance model of broad externalizing (ΔCFI=.000; ΔNCI=.001),
indicating that partial strong longitudinal measurement invariance was
supported. Given that full strong invariance was not supported, we did
not test for strict measurement invariance.

\emph{Narrow Externalizing Factor}. Model fit was not significantly
impacted by adding constraints for weak, strong, or strict measurement
invariance. Thus, we were able to demonstrate strict longitudinal
measurement invariance for the narrow externalizing factor.

\emph{Irritability}. Model fit was not significantly impacted by adding
constraints for weak or strong measurement invariance. However, the
constraints for strict invariance resulted in a significant decrement in
model fit (ΔCFI=.003; ΔNCI=.009). Thus, we were only able to demonstrate
strong longitudinal measurement invariance.

\emph{Neurodevelopmental Problems}. Relative to the configural model,
model fit was not significantly impacted after constraining factor
loadings to be equal (ΔCFI=.002; ΔNCI=.005), indicating weak measurement
invariance. However, after imposing constraints on the intercepts of the
parcel indicators (strong invariance), model fit was significantly
negatively impacted (ΔCFI=.006; ΔNCI=.018). Based on modification
indices, the intercepts of parcels 3 and 4 were freed at baseline and
three-year follow-up. The resulting partial strong measurement
invariance model showed excellent overall fit (χ\^{}2=524.25; df=88;
RMSEA=.020; CFI=.993; TLI=.991) and no decrement in model fit compared
to the weak invariance model (ΔCFI=.002; ΔNCI=.005), indicating that
partial strong longitudinal measurement invariance was supported.

\textbf{Aim 2b: Effect Sizes for Longitudinal Measurement Noninvariance}
\textbf{Table 3} provides results for dMACS and additional effect sizes
(i.e., proportion of mean change in the latent factor attributable to
noninvariance) for the broad externalizing factor and neurodevelopmental
problems factor (i.e., the two factors for which noninvariance was
detected). Results showed that the effect sizes of indicator
noninvariance were small, with dMACS ranging from .00-.18. While dMACS
summarized the magnitude of noninvariance at the indicator level, there
was a more noticeable impact of noninvariance at the latent level of the
broad externalizing dimension, with noninvariance accounting for between
25-30\% of observed mean change in the dimension across the pairwise
comparisons.

\textbf{Sensitivity Analyses} We conducted a series of sensitivity
analyses to examine the robustness of our results relative to other
analytic choices. This included 1) examining alternative factoring and
rotation methods for our bass-ackwards analysis, 2) exploring the impact
of variability in item-parcel allocation on measurement model parameters
(i.e., loadings) and model fit statistics, and 3) examining the impact
of cluster effects of family and study site in the ABCD study. Overall,
sensitivity analyses showed that all results were robust to these
factors. An overview of these analyses and results are provided in
Supplementary Materials (see \textbf{Section II} and \textbf{Table S6}).

\textbf{Discussion} The present study sought to delineate the
hierarchical structure of externalizing symptoms in the ABCD study, and
evaluate the longitudinal measurement invariance of the identified
dimensions. Overall, there was mixed support for our preregistered
hypotheses. Specifically, in our bass-ackwards analysis, we did not find
evidence for easily interpretable disinhibition or antagonism dimensions
(\emph{Hypothesis 1a}) and instead identified three factors: narrow
externalizing, irritability, and neurodevelopmental problems. There was
support for the discriminant validity of the specific externalizing
dimensions (\emph{Hypothesis 1b}) and, consistent with our hypothesis
(\emph{Hypothesis 2}), strong measurement invariance was established for
two of these factors (narrow externalizing and irritability). While
strong invariance could not be established for the broad externalizing
and neurodevelopmental problems factors, a less restrictive model
(partial strong invariance) was supported. Importantly, effect size
metrics for longitudinal invariance highlighted the necessity of
examining the impact of noninvariance at the level of indicators and
latent dimensions, as noninvariance showed the greatest impact on the
latent level here. Together, results have implications for extending the
HiTOP model to youth and for future research on the development of
externalizing problems in the ABCD study as well as other prospective,
longitudinal studies.

\textbf{Hierarchical Structure of Externalizing Psychopathology: Links
to the HiTOP Model} While the current study identified a broad
externalizing dimension that strongly aligns with the externalizing
superspectrum of HiTOP \citep{krueger2021}, the specific externalizing
dimensions showed greater divergence from what has been found in adults.
Contrary to hypotheses, clear disinhibition or antagonism dimensions
were not identified. Instead, three dimensions with notable links to
prior work and some similarities to the HiTOP model emerged. The first
was a narrow externalizing dimension, consisting of aggression, meanness
to others, and rule breaking behaviors. This factor mirrored antagonism
dimensions identified in similar HiTOP-focused analyses in youth
samples, which included items associated with aggressive behavior and
meanness toward others \citep{forbes2023}. In our sample, this dimension
also contained content aligned with rule-breaking behavior (e.g.,
stealing, vandalism), which stands somewhat in contrast to past research
highlighting the importance of distinguishing rule-breaking behaviors
from more overtly antagonistic behaviors like aggression
\citep{burt2012}. Interestingly, even at subsequent levels of the
bass-ackwards hierarchy, this narrow externalizing factor did not split
into more more specific antagonistic (e.g., aggression, callousness) or
rule-breaking factors, suggesting that these features could not be
empirically separated. While this may be related to developmental timing
and/or the nature of this community sample, it will be important to
continue examining the hierarchical structure of externalizing pathology
across development, and empirically determine if and when these
dimensions differentiate in the ABCD Study.

Results also identified an irritability dimension, characterized by
affective reactivity and lability, hostility, and distrust of others.
While not predicted, this dimension echoes prior work documenting the
central role of irritability in early manifestations of externalizing
behavior in youth \citep{leibenluft2013}. Notably, characteristics of
the irritability dimension found here -- hostility and distrust of
others -- are similar to features of the antagonistic externalizing
dimension that has been identified in adult samples
\citep{mullins-sweatt2022}. However, this dimension was also
characterized by a general moodiness that is not explicitly
interpersonal in nature, and may point to affective lability as a
developmental precursor to interpersonal hostility. In addition, extant
research shows that irritability is an important subdimension of
externalizing behaviors in youth that predicts subsequent internalizing
psychopathology \citep[e.g.,][]{burke2012}. Moreover, irritability is a
transdiagnostic factor observed in an array of psychiatric disorders,
and can help to explain the co-occurrence of externalizing and
internalizing problems \citep{finlay-jones2023}. Thus, irritability
represents an important dimension within developmental models of
externalizing behavior and extensions of the HiTOP to youth.

A neurodevelopmental problems dimension defined by inattention and
hyperactivity emerged and remained intact across all levels of our
bass-ackwards hierarchy. Notably, much of the factor content aligns with
ADHD symptoms, and past structural work in the ABCD study has labeled
related factors ``inattention'' or ``ADHD'' factors
\citep[e.g.,][]{clark2023}. Nonetheless, the presence of such a factor
echoes prior work on the structure of psychopathology in youth
\citep{michelini2019, holmes2021} and underscores the importance of this
dimension in developmental models of externalizing behavior. Moreover,
the emergence of this dimension dovetails with other research
emphasizing the importance of understanding neurodevelopmental problems
as a transdiagnostic dimension in youth psychopathology
\citep{astle2022}. Past research has also emphasized the importance of
neurodevelopmental problems in the onset and persistence of antisocial
behavior \citep{goozen2022}, which further contextualizes the presence
of neurodevelopmental problems within the domain of youth externalizing
problems.

\textbf{Evidence of Discriminant Validity for Externalizing Dimensions}
The three identified externalizing dimensions demonstrated evidence of
discriminant validity based on their bivariate correlations. This was
most notable for parent-reported diagnostic symptom counts and prosocial
behavior, which may be unsurprising given shared method variance. The
largest differences between dimensions were observed for disorders that
are typically conceptualized as externalizing, with the narrow
externalizing factor demonstrating the strongest association with CD
symptoms, the irritability factor with ODD symptoms, and the
neurodevelopmental problems factor with ADHD symptoms.

Controlling for overlap among the dimensions generally enhanced this
pattern of discriminant validity, but several results are noteworthy.
First, the unique variance in the narrow externalizing dimension showed
the smallest relations to internalizing symptoms and was negatively
related to anxiety disorder symptoms, while maintaining similar
relations with conduct disorder symptoms and prosocial behavior. Thus,
the unique aspects of this dimension appear to be aligned with
callous-unemotional features---past research has demonstrated how these
features are weakly related to broad internalizing problems
\citep{cardinale2020} and demonstrate negative relations with with
anxiety specifically \citep[e.g.,][]{frick1999}. Second, the
irritability dimension maintained the strongest associations with mood
and anxiety symptoms even after accounting for overlap between factors,
echoing research pointing to irritability as a transdiagnostic indicator
of psychopathology \citep{finlay-jones2023}. Finally, the unique
variance in neurodevelopmental problems was unrelated to CD, and ODD
symptoms as well as prosociality. This finding has implications for
research on ``cool'' (abstract-cognitive) and ``hot'' (reward-related)
executive functioning systems, and suggests that the unique variance in
the neurodevelopmental problems dimension appears to largely reflect the
``cold'' end of the hot-cold continuum in executive function
\citep{zelazo2020}.

\textbf{Longitudinal Measurement Invariance for Externalizing
Dimensions} Tests of longitudinal measurement invariance showed that
strict invariance held for the narrow externalizing dimension and strong
invariance held for the irritability dimension. Statistically, strong
invariance indicates that all longitudinal changes in the means and
covariances of the indicators can be attributed to changes in their
respective latent dimensions across time \citep{grimm2017}.
Substantively, strong invariance indicates that researchers can conduct
comparisons across time for these externalizing dimensions and
investigate their associations with other variables without worrying
that the indicators have been systematically influenced by unmeasured
factors \citep{sidaman2010}. Notably, partial strong invariance was met
for the broad externalizing and neurodevelopmental problems factors,
indicating that longitudinal changes in observed indicator means and
covariances are attributable to changes in the latent means, \emph{but
only for the longitudinally invariant indicators}. Given that the
neurodevelopmental problems factor, in particular, focuses on
developmentally specific content that likely manifests differently as
youth mature, some degree of noninvariance is unsurprising
\citep{petersen2020}. Though strong invariance is the typical benchmark
that must be met before making mean comparisons across time, researchers
commonly use partial strong invariance models of externalizing
\citep[e.g.,][]{king2018} and simulation-based investigations have found
that using partially invariant models for subsequent analyses does not
result in biased parameters in common applied settings \citep{shi2019}.
Taken together, findings provide empirical support for the examination
of between- and within-person change in these latent externalizing
dimensions over time.

Additionally, our investigation of invariance effect sizes, while not
commonly assessed or reported, points to important additional areas of
consideration. First, all \emph{dMACS} for the broad externalizing and
neurodevelopmental problem models (i.e., the two models where
noninvariant indicators were detected) fell below our preregistered
threshold for inferring meaningful impact of measurement invariance
(\emph{dMACS} \textless{} .20). This suggests that even though our
previous tests indicated that constraining intercepts significantly
worsened model fit, the difference in indicator intercepts across time
was very small --- providing further support for the notion that our
findings of only partial strong invariance for certain factors do not
preclude their use in subsequent analyses. Second, despite the
consistently small \emph{dMACS} effect sizes, there was a noteworthy
impact of noninvariance but only at the latent level of broad
externalizing. Specifically, between 25-30\% of observed mean
differences were due to measurement invariance. Substantively, this
suggests that comparisons between the broad externalizing dimension at
baseline and later time points will overestimate mean differences if
this bias is not taken into account. These results highlight a
well-known, but infrequently examined, implication of measurement
noninvariance \citep{vandenberg2000}---that the cumulative effects of
noninvariant indicators can lead to larger biasing effects at the latent
level (observed for broad externalizing), or in other cases, indicators
can cancel out, leading to almost no bias at the latent level (observed
for neurodevelopmental problems). Results underscore the importance of
examining the effect of longitudinal noninvariance at the latent
\emph{and} indicator level \citep{clark2021}. Future work examining the
latent broad externalizing factor over time would need to consider the
biasing impact of noninvariance. For other externalizing dimensions, the
impact of noninvariance was trivial (neurodevelopmental problems) or
there was evidence in support of strong measurement invariance (narrow
externalizing, irritability), suggesting the examination of their
trajectories will reflect true changes and not measurement artifacts.

\textbf{Limitations} While the current study addresses notable gaps in
the literature, results should be considered in light of several
noteworthy limitations. First, as is the case with any factor analytic
technique, the solution is dependent on the input \citep{fabrigar1999}.
That is, our results assume that the CBCL items provide sufficient
coverage of the externalizing dimension. While the CBCL is
well-validated and extensively used in developmental samples,
insufficient coverage of any specific content (e.g., lying) and/or low
base rates of less developmentally salient behaviors (e.g., substance
use) limit our ability to identify such dimensions. This point is
important to keep in mind when drawing comparisons with other studies
that have utilized different measures of externalizing behaviors
\citep[e.g.,][]{forbes2023} as they may differ in content coverage that
generate divergent factor solutions, particularly at more fine-grained
levels.

Second, externalizing dimensions were assessed using a single informant
(i.e., parent-report) and it is perhaps unsurprising that associations
with external clinical correlates were strongest for parent-reported
indices. Given that cross-informant reliability for externalizing
psychopathology in youth is consistently low \citep{reyes2004}, it will
be particularly important to examine the structure of externalizing
dimensions in the ABCD study using alternative informants (e.g., youth,
teacher). Additionally, expanding beyond monomethod assessments and
incorporating multiple levels of analysis to more precisely
conceptualize psychopathology constructs dimensions will be important
\citep[e.g.,][]{joyner2023}.

Finally, while the present study made use of recommended best-practices
when using parcels \citep[see][]{sterba2023}, including a transparent
preregistration, empirically identified unidimensional factors, and
inclusion of sensitivity analyses examining the impact of item-parcel
allocation variability, the use of parcels has well-documented
limitations. For example, their use necessarily overlooks fine-grained
psychometric information about specific items. Though our focus was on
latent externalizing dimensions and not item psychometrics per se,
item-level characteristics are valuable to understand and future work
could aim to better assess this and further refine the assessment of
externalizing dimensions.

\textbf{Future Directions} An important goal is to further expand
investigations into the HiTOP in youth to better understand the
lower-order structure of psychopathology symptoms. Our findings are in
line with emerging evidence suggesting that the broad externalizing
dimension of HiTOP is compatible across adults and youth
\citep[e.g.,][]{michelini2019, forbes2023}; however, organization of
finer-grained dimensions is less consistent. A valuable question for
further research would be to examine the replicability of the
finer-grained dimensions of externalizing symptoms identified here,
using alternative assessments, informants, and developmental timepoints.
Future research can also examine other factors relevant to measurement
invariance (e.g., sex, ethnicity) using the measurement models developed
in the present study. However, subgroup analyses using the parcel
developed in the present study would necessitate further sensitivity
analyses, as within-sample item-parcel allocation variability is
proportional to sampling error \citep{sterba2010}, meaning that our use
of the total sample in the ABCD study helped minimize item-parcel
allocation variability.

Last, future work can also examine alternative hierarchical modeling
frameworks. Our bass-ackwards approach can be contrasted with
higher-order factor models estimated from ABCD study data
\citep[e.g.,][]{clark2023} in terms of how variance is partitioned---in
higher-order factor models, each level of the hierarchy is based on
different variances. For example, the broad externalizing dimension in a
higher-order model would represent the shared variance among narrow
externalizing, irritability, and neurodevelopmental problems dimensions.
In the bass-ackwards approach, the broad externalizing factor reflects
the shared variance of all CBCL items. These approaches provide
different information and future work can expand on the present results
by investigating the use of higher-order factor models.


  \bibliography{references.bib}


\end{document}
